\documentclass[a4paper,10pt]{article}
\usepackage{hyperref}
\usepackage[utf8]{inputenc}
\usepackage{listings}
\usepackage{lmodern}
\usepackage[T1]{fontenc}
\usepackage[utf8]{inputenc}
\usepackage{indentfirst}
\usepackage{color}
\usepackage{graphicx}
\usepackage{microtype}
\usepackage{enumerate}
\usepackage{lipsum}
\usepackage[brazilian,hyperpageref]{backref}
\usepackage[alf]{abntex2cite}
\usepackage{float}
\usepackage{amsmath}

\graphicspath{ {pics/} }
\lstset{ %
  backgroundcolor=\color{white},   % choose the background color; you must add \usepackage{color} or \usepackage{xcolor}
  basicstyle=\footnotesize,        % the size of the fonts that are used for the code
  breakatwhitespace=false,         % sets if automatic breaks should only happen at whitespace
  breaklines=true,                 % sets automatic line breaking
  captionpos=b,                    % sets the caption-position to bottom
  commentstyle=\color{mygreen},    % comment style
  extendedchars=true,              % lets you use non-ASCII characters; for 8-bits encodings only, does not work with UTF-8
  frame=single,	                   % adds a frame around the code
  keywordstyle=\color{blue},       % keyword style
  numberstyle=\tiny\color{mygray}, % the style that is used for the line-numbers
  rulecolor=\color{black},         % if not set, the frame-color may be changed on line-breaks within not-black text (e.g. comments (green here))
  stepnumber=2,                    % the step between two line-numbers. If it's 1, each line will be numbered
  tabsize=2,	                   % sets default tabsize to 2 spaces
  title=\lstname                   % show the filename of files included with \lstinputlisting; also try caption instead of title
}

%opening
\title{Problem Set 01 - Report}
\author{Felipe Duarte dos Reis}

\begin{document}

\maketitle

\begin{abstract}
Report for the problem set 01, assigned in Computer Vision class. All exercises were solved in Python language using the
version 2.4.
\end{abstract}

\section{Problem 1}
Here we discuss the conclusions about exercises involving measures in a window over an image. 
The solution is shown in \autoref{lst:ex1} in \autoref{sec:listings}. \\
We could say that an window $W_p$ is over an homogeneous area of images values when the standard deviation of this window is closer to 
zero. And, in the other hand, an inhomogeneous area under a window $W_p$ have the standard deviation $\sigma$ far from zero.
We can also look to the standard error, given by \autoref{eq1}.

\begin{equation}
\label{eq1}
  \epsilon = \frac{\sigma}{\sqrt{n}}
\end{equation}

If $\epsilon > 1$ we can see diversity of vlaues $u$ inside $W_p$. 
When $\epsilon < 1$ the values $u$ inside $W_p$ are similar and the window isover an homogeneous area.
In the \autoref{fig:img1}, in the left we are seeing the window located under the piano's keyboard, where is shadowed.
In this window we have just values next to black, and the error of
this window is small compared to the window located next the piano's foot where we have values next to white, black and golden.

\begin{figure}[H]
\centering
\caption{Up: Inhomogeneous area. Down: Homogeneous area}
\label{fig:img1}
\includegraphics[width=200px]{img1}
\end{figure}


\section{Problem 2}
Here we discuss the conclusions about exercises involving data measures in frames sequenced over time. 
The solution is shown in \autoref{lst:ex2} in \autoref{sec:listings}. \\
We recorded a little video to problem two, and reduced the resulution and frame rate. 
After reading 50 frames we compute three data measures: contrast, mean and variance. 
We have chosen the constrast  to normalize the other data measurments.
Using the L1-metric (mean of absolute diferences between functions), we have the values described in \autoref{tab1}:

\begin{table}[H]
  \centering
  \caption{Distances between data measures}
  \label{tab1}
  \begin{tabular}{ll}
    Measures	&	Distances      \\
    CONTRAST, MEAN	&	0.712764534377 \\
    CONTRAST, STD	&	0.586346704051 \\
    MEAN, STD	&	0.23814664864  
  \end{tabular}
\end{table}

\section{Problem 3}
Here we discuss the conclusions about exercises involving the Fourier Transform and Inverse Fourier Transform. \\
\subsection{Item a}
The solution is shown in \autoref{lst:ex3a} in \autoref{sec:listings}. \\
The predominant component is clear the module as shown in \autoref{fig:img2}. 
The top pictures are the original photographies used in this item.
\begin{figure}[H]
 \centering
  \includegraphics[width=400px]{img2}
  \caption{In top of image we see original photos, left-down the combination of mug's magnitude with book's phase. The oposite in right-down.}
 \label{fig:img2}
\end{figure}

The left-down picture shows the combination of mug's magnitude component and book's phase component, and right-down picture 
shows the combination of book's magnitude component and mug's phase component.

\subsection{Item b}
The solution is shown in \autoref{lst:ex3b} in \autoref{sec:listings}. \\
When you increase magnitudes you decrease the intensity values in the spatial domain. 
When you rotate phase angles $\frac{\pi}{4}$ you decrease the magnitude of image to zero. 
And if you continue rotating the phase you will increase the magnitude until a maxima value in spatial domain, namely 255.

\section{Problem 5}
Here we discuss the conclusions about exercises involving sigma filter and histogram equalization. 
The solution is shown in \autoref{lst:ex5} in \autoref{sec:listings}. \\\
We applied a sigma filter discussed in the \cite[p. 58]{Klette:Concise_Computer_Vision} in the \autoref{fig:img4}, 
and then we calculate the histogram equalization following the \autoref{eq2} varying $r$ parameter from 0 to 1.9.

\begin{equation}
 \label{eq2}
 g^{r}(u) = \frac{G_{max}}{Q}\sum_{w=0}^{u}h_I(w)^r  \text{ with }
 Q = \sum_{w=0}^{G_{max}} h_I(w)^r
\end{equation}

\begin{figure}[H]
 \centering
 \caption{Noisy picture used in problem 5}
 \label{fig:img4}
 \includegraphics{img4}
\end{figure}

The histogram surface plot is shown in \autoref{fig:img5}. We can visualize, as said in the statement, 
that over the $r = 1$ there is a quasi-normal curve. 
For $r > 1$ the histogram values are more homogeneous, and for $r < 1$ there a clear tendency 
for higher values in grayscale (higher frequencies between 140 and 180).

\begin{figure}[H]
 \centering
 \caption{Histogram surface varying $r$ from 0 to 1.9}
 \label{fig:img5}
 \includegraphics[width=400px]{img5}
\end{figure}

\section{Problem 6}
In this section we discuss the conclusions about exercise involving edge detection using directional derivatives. 
The solution is shown in \autoref{lst:ex6} in \autoref{sec:listings}. \\\
We used here the same piano image shown in \autoref{fig:img1}. We developed a filter kernel that takes the derivative
in xy axis in the original position and rotated $$\frac{\pi}{4}$$ rad. We used to the Sobel and Laplacian operator
over the same image to compare the results. This images are shown in \autoref{fig:img6}.

\begin{figure}[H]
 \centering
 \caption{Left: Results of derivatives in different directions. Middle: Sobel operator. Right: Laplacian operator}
 \label{fig:img6}
 \includegraphics[width=400px]{img6}
\end{figure}
Our convolution is not the best result, of course. It can detect the most relevant edges, where we se in the original image a gratter
diference between grayscale values. But it still identfy the intensity variations inside the edges, where the diference is small and
shouldn't be detected. In Sobel and Laplacian operator the signifcant edges appears clearly. But the Laplacian in this test highlights 
the outerbrder better than Sobel.

\appendix
\section{Listings}
\label{sec:listings}
\lstinputlisting[language=Python, caption={Solution of problem one}, label={lst:ex1}]{../ex1/ex1.py}
\lstinputlisting[language=Python, caption={Solution of problem two}, label={lst:ex2}]{../ex2/ex2.py}
\lstinputlisting[language=Python, caption={Solution of problem three, item a}, label={lst:ex3a}]{../ex3/ex3a.py}
\lstinputlisting[language=Python, caption={Solution of problem three, item b}, label={lst:ex3b}]{../ex3/ex3b.py}
\lstinputlisting[language=Python, caption={Solution of problem five}, label={lst:ex5}]{../ex5/ex5.py}

\bibliography{references.bib}

\end{document}
